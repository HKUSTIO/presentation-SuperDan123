\documentclass[aspectratio=169]{beamer}  % 16:9 aspect ratio

% Use a clean theme as base
\usetheme{default}
\usecolortheme{default}

% Custom colors from HKUST logo
\definecolor{hkustblue}{RGB}{0, 51, 119}    % Navy blue from logo
\definecolor{hkustgold}{RGB}{180, 141, 61}  % Golden brown from logo
\definecolor{lightgray}{RGB}{236, 240, 241}

% Customize the appearance
\setbeamercolor{structure}{fg=hkustblue}
\setbeamercolor{background canvas}{bg=white}
\setbeamercolor{normal text}{fg=hkustblue}
\setbeamercolor{frametitle}{fg=hkustblue,bg=white}
\setbeamercolor{itemize item}{fg=hkustgold}
\setbeamercolor{itemize subitem}{fg=hkustgold}
\setbeamercolor{block title}{fg=white,bg=hkustblue}
\setbeamercolor{block body}{fg=hkustblue,bg=lightgray}
\setbeamercolor{title}{fg=hkustblue}
\setbeamercolor{subtitle}{fg=hkustgold}

% Remove navigation symbols
\setbeamertemplate{navigation symbols}{}

% Customize frame title
\setbeamertemplate{frametitle}{
    \vspace*{0.5cm}
    \insertframetitle
    \vspace*{0.3cm}
    \begin{beamercolorbox}[wd=\paperwidth,ht=0.2pt]{structure}
    \end{beamercolorbox}
}

% Customize itemize bullets
\setbeamertemplate{itemize item}{\small\raise0.5pt\hbox{\textbullet}}
\setbeamertemplate{itemize subitem}{\tiny\raise1.5pt\hbox{\textbullet}}

% Packages
\usepackage{graphicx}
\usepackage{amsmath}
\usepackage{hyperref}
\usepackage{ragged2e}
% \usepackage{newtxtext} 
% \usepackage{newtxmath}
\AtBeginEnvironment{frame}{\justifying}
% Title page information
\title{Two-sided Markets, Pricing, and Network Effects}
\subtitle{Theoretical and Empirical Study}
\author{Dan (Daisy) Song \quad Jing You}
\institute{Hong Kong University of Science and Technology}
\date{\today}

\begin{document}

% Title page
\begin{frame}
    \titlepage
\end{frame}

% Table of contents
\begin{frame}{Outline}
    \tableofcontents
\end{frame}

% Section 1
\section{Introduction}
\begin{frame}{Introduction}
    \begin{itemize}
        \item A central aspect of platform is the role of network effects:
        \vspace{1em}
        \begin{itemize}
            \item \textbf{Network effect} refers to any situation in which the value of a product, service, or platform depends on the number of buyers, sellers, or users who leverage it
            \vspace{1em}
            \item \textbf{Direct network effects} occur when the value of a product, service, or platform increases simply because the number of users increases, causing the network itself to grow
            \vspace{1em}
            \item \textbf{Indirect network effects} occur when a platform or service depends on two or more user groups, such as producers and consumers, buyers and sellers, or users and developers
        \end{itemize}

    \end{itemize}
\end{frame}

\begin{frame}{Introduction}
    \begin{itemize}        
        \item Indirect network leads to feedback loops between the two sides of the market, thus increasing efficiencies and also potential market power
        \vspace{1em}
        \item Pricing is one of the key tools that platforms can manage to success
    \end{itemize}
    \vspace{2em}

    \textcolor{red}{Main Focus}: \textbf{indirect network effects and pricing strategies in two-sided markets}
\end{frame}

\begin{frame}{Introduction}
    \begin{itemize}        
        \item Two-sided market: at least two distinct sets of agents interact through an intermediary and in which the behavior of each set of agents directly impacts the utility, or the profit of the other agents
        \vspace{1em}
        \item \textbf{Two-sided market} vs \textbf{Two-sided strategies}
        \vspace{1em}
        \item Treating two-sidedness as a market concept
    \end{itemize}
    \vspace{2em}
\quad \quad Now let's proceed to the benchmark model in Armstrong(2006)
\end{frame}

\section{Benchmark Model}
\begin{frame}{Armstrong(2006): Monopoly Pricing}
    \begin{equation*}
        \Large
        u_{j}^{i} = \alpha_{j}^{i} n^{i} + \xi_{j}^{i}
    \end{equation*}
    \begin{block}{Basic Setting}
        \begin{enumerate}
            \item Two groups of agents: j = 1, 2
            \item Only one platform in the market: i = 1
            \item No fixed benefits from using the platform: $\xi_{j}^{i} = 0$
            \item One group cares about the number of agents in the other group but not itself
            \item Interaction benefits are different across groups: $\alpha_{j}^{i} \neq \alpha_{k}^{i}$
            \item The platform can charge a per-agent fee to each group of agents
        \end{enumerate}
    \end{block}

\end{frame}

\begin{frame}{Armstrong(2006): Monopoly Pricing}
    \textbf{Utility of agents}
    \vspace{1em}
    \begin{equation}
        \large
        u_{1} = \alpha_{1} n_{2} - p_{1}
            \qquad
        u_{2} = \alpha_{2} n_{1} - p_{2}
    \end{equation}
    \begin{itemize}
      \item $\alpha_{1}$: Interaction benefit a group-1 agent gets from each group-2 agent
      \item $\alpha_{2}$: Interaction benefit a group-2 agent gets from each group-1 agent
      \item $n_{1}$: number of group-1 agents on the platform
      \item $n_{2}$: number of group-2 agents on the platform
      \item $p_{1}, p_{2}$: prices charged by the platform to groups 1 and 2
    \end{itemize}
\end{frame}

\begin{frame}{Armstrong (2006): Monopoly Pricing}
    \textbf{Demand Functions}
    \vspace{1em}
    
    \begin{equation}
        \large
        n_{1} = \phi_{1}(u_{1})
        \qquad
        n_{2} = \phi_{2}(u_{2})
    \end{equation}
    
    \begin{itemize}
      \item $n_{1}, n_{2}$: numbers of agents from groups 1 and 2 
                            who join the platform
      \item $\phi_{1}, \phi_{2}$: increasing demand functions mapping 
                                  $u_{1}, u_{2}$ (utilities) to participation
      \item $u_{1}, u_{2}$: utilities offered to agents in groups 1 and 2
    \end{itemize}
\end{frame}

\begin{frame}{Armstrong (2006): Monopoly Pricing}
    \textbf{Profit Function}
    \vspace{1em}

    \begin{equation}
        \large
        \pi(u_{1}, u_{2}) \;=\; 
            \phi_{1}(u_{1})\Bigl[\alpha_{1}\,\phi_{2}(u_{2}) - u_{1} - f_{1}\Bigr]
            \;+\;
            \phi_{2}(u_{2})\Bigl[\alpha_{2}\,\phi_{1}(u_{1}) - u_{2} - f_{2}\Bigr]
    \end{equation}
    
    \begin{itemize}
      \item $\pi(u_{1},u_{2})$: platform’s profit in terms of utilities
      \item $f_{1},f_{2}$: per‐agent costs for groups 1 and 2
      \item $\alpha_{1}, \alpha_{2}$: cross‐group interaction benefits
      \item $\phi_{1}, \phi_{2}$: same demand functions as in the previous slide
    \end{itemize}
\end{frame}

\begin{frame}{Armstrong (2006): Monopoly Pricing}
    \justifying

    Let the aggregate consumer surplus of group \(i = 1, 2\) be \(v_i(u_i)\), 
    where \(v_i(\cdot)\) satisfies the envelope condition 
    \( v_i'(u_i) \equiv \phi_i(u_i) \). Then welfare, as measured by the unweighted 
    sum of profit and consumer surplus, is
    \[
       w = \pi(u_1, u_2) \;+\; v_1(u_1) \;+\; v_2(u_2).
    \]
    
    It is easily verified that the welfare‐maximizing outcome has the utilities
    \[
       u_1 = (\alpha_1 + \alpha_2)\,n_2 \;-\; f_1,
       \qquad
       u_2 = (\alpha_1 + \alpha_2)\,n_1 \;-\; f_2.
    \]
    
    From expression (1), the socially optimal prices satisfy
    \[
       p_1 = f_1 \;-\; \alpha_2\,n_2,
       \qquad
       p_2 = f_2 \;-\; \alpha_1\,n_1.
    \]
\end{frame}

\begin{frame}{Armstrong (2006): Monopoly Pricing}
    \justifying

    From expression~(2), the profit‐maximizing prices satisfy:
    \[
      p_{1} 
      \;=\; f_{1} \;-\; \alpha_{2}\,n_{2}
              \;+\; \frac{\phi_{1}(u_{1})}{\phi_{1}'(u_{1})},
      \quad
      p_{2} 
      \;=\; f_{2} \;-\; \alpha_{1}\,n_{1}
              \;+\; \frac{\phi_{2}(u_{2})}{\phi_{2}'(u_{2})}.
    \]
\end{frame}

\begin{frame}{Proposition 1}
    \justifying  % Requires \usepackage{ragged2e} in your preamble, if desired
    
    \textbf{Proposition 1.} Write
    \[
      \eta_{1}\bigl(p_{1} \mid n_{2}\bigr) 
      \;=\; 
      \frac{p_{1}\,\phi_{1}'(\alpha_{1}n_{2} - p_{1})}{\phi_{1}(\alpha_{1}n_{2} - p_{1})},
      \quad
      \eta_{2}\bigl(p_{2} \mid n_{1}\bigr) 
      \;=\;
      \frac{p_{2}\,\phi_{2}'(\alpha_{2}n_{1} - p_{2})}{\phi_{2}(\alpha_{2}n_{1} - p_{2})}.
    \]
    
    for a group’s price elasticity of demand given the other group’s participation level.  
    Then the profit‐maximizing pair of prices satisfy
    
    \[
      \frac{p_{1} \;-\; \bigl(f_{1} - \alpha_{2}n_{2}\bigr)}{p_{1}}
      \;=\;
      \frac{1}{\eta_{1}(p_{1} \mid n_{2})},
      \quad
      \frac{p_{2} \;-\; \bigl(f_{2} - \alpha_{1}n_{1}\bigr)}{p_{2}}
      \;=\;
      \frac{1}{\eta_{2}(p_{2} \mid n_{1})}.
    \]
    
    \end{frame}

\begin{frame}{Armstrong (2006): Main Conclusions of Monopoly Pricing}
    \justifying  % Requires \usepackage{ragged2e} in the preamble
    
    A monopolist platform, serving two distinct groups that benefit from interacting 
    with one another, does not simply set each group’s price to cover the marginal 
    cost of serving that group. Instead, it balances the prices charged to each side 
    to account for cross‐group externalities and differences in demand elasticities.
    This leads to the following key insights:
    
    \begin{itemize}
        \item \textbf{Cross‐Group Externalities:} Prices on one side depend on how 
        that side’s participation affects the other side’s willingness to join.
    
        \item \textbf{Balancing Demand:} If one side’s presence is highly valuable 
        to the other side, the platform may lower or subsidize that side’s fee 
        to maximize overall profit.
    
        \item \textbf{Interdependence of Pricing:} Optimal monopoly prices for each 
        group differ from the standard one‐sided monopoly outcome, reflecting 
        the platform’s internalization of cross‐group effects.
    \end{itemize}
    
\end{frame}
    
\begin{frame}{Armstrong(2006): Competition for the Market}
    \begin{block}{Important Block}
        Key information goes here
    \end{block}
    
    \begin{itemize}
        \item Method 1
        \item Method 2
    \end{itemize}
\end{frame}

\begin{frame}{Armstrong(2006): Competition on the Market}
    \begin{block}{Important Block}
        Key information goes here
    \end{block}
    
    \begin{itemize}
        \item Method 1
        \item Method 2
    \end{itemize}
\end{frame}

% Section 3
\section{Comments and Insights}
\begin{frame}{Results}
    \begin{columns}
        \column{0.5\textwidth}
        \begin{itemize}
            \item Result 1
            \item Result 2
        \end{itemize}
        
        \column{0.5\textwidth}
        % Placeholder for a figure
        \centering
        [Your figure here]
    \end{columns}
\end{frame}

% Section 4
\section{Summary}
\begin{frame}{Conclusion}
    \begin{itemize}
        \item Main finding 1
        \item Main finding 2
        \item Future work
    \end{itemize}
\end{frame}

\end{document} 