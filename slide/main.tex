\documentclass[aspectratio=169]{beamer}  % 16:9 aspect ratio

% Use a clean theme as base
\usetheme{default}
\usecolortheme{default}

% Custom colors from HKUST logo
\definecolor{hkustblue}{RGB}{0, 51, 119}    % Navy blue from logo
\definecolor{hkustgold}{RGB}{180, 141, 61}  % Golden brown from logo
\definecolor{lightgray}{RGB}{236, 240, 241}

% Customize the appearance
\setbeamercolor{structure}{fg=hkustblue}
\setbeamercolor{background canvas}{bg=white}
\setbeamercolor{normal text}{fg=hkustblue}
\setbeamercolor{frametitle}{fg=hkustblue,bg=white}
\setbeamercolor{itemize item}{fg=hkustgold}
\setbeamercolor{itemize subitem}{fg=hkustgold}
\setbeamercolor{block title}{fg=white,bg=hkustblue}
\setbeamercolor{block body}{fg=hkustblue,bg=lightgray}
\setbeamercolor{title}{fg=hkustblue}
\setbeamercolor{subtitle}{fg=hkustgold}

% Remove navigation symbols
\setbeamertemplate{navigation symbols}{}

% Customize frame title
\setbeamertemplate{frametitle}{
    \vspace*{0.5cm}
    \insertframetitle
    \vspace*{0.3cm}
    \begin{beamercolorbox}[wd=\paperwidth,ht=0.2pt]{structure}
    \end{beamercolorbox}
}

% Customize itemize bullets
\setbeamertemplate{itemize item}{\small\raise0.5pt\hbox{\textbullet}}
\setbeamertemplate{itemize subitem}{\tiny\raise1.5pt\hbox{\textbullet}}

% Packages
\usepackage{graphicx}
\usepackage{amsmath}
\usepackage{hyperref}
\usepackage{ragged2e}
% \usepackage{newtxtext} 
% \usepackage{newtxmath}
\AtBeginEnvironment{frame}{\justifying}
% Title page information
\title{Two-sided Markets, Pricing, and Network Effects}
\subtitle{Theoretical and Empirical Study}
\author{Dan (Daisy) Song \quad Jing You}
\institute{Hong Kong University of Science and Technology}
\date{\today}

\begin{document}

% Title page
\begin{frame}
    \titlepage
\end{frame}

% Table of contents
\begin{frame}{Outline}
    \tableofcontents
\end{frame}

% Section 1
\section{Introduction}
\begin{frame}{Introduction}
    \begin{itemize}
        \item A central aspect of platform is the role of network effects:
        \vspace{1em}
        \begin{itemize}
            \item \textbf{Network effect} refers to any situation in which the value of a product, service, or platform depends on the number of buyers, sellers, or users who leverage it
            \vspace{1em}
            \item \textbf{Direct network effects} occur when the value of a product, service, or platform increases simply because the number of users increases, causing the network itself to grow
            \vspace{1em}
            \item \textbf{Indirect network effects} occur when a platform or service depends on two or more user groups, such as producers and consumers, buyers and sellers, or users and developers
        \end{itemize}

    \end{itemize}
\end{frame}

\begin{frame}{Introduction}
    \begin{itemize}        
        \item Indirect network leads to feedback loops between the two sides of the market, thus increasing efficiencies and also potential market power
        \vspace{1em}
        \item Pricing is one of the key tools that platforms can manage to success
    \end{itemize}
    \vspace{2em}

    \textcolor{red}{Main Focus}: \textbf{indirect network effects and pricing strategies in two-sided markets}
\end{frame}

\begin{frame}{Introduction}
    \begin{itemize}        
        \item Two-sided market: at least two distinct sets of agents interact through an intermediary and in which the behavior of each set of agents directly impacts the utility, or the profit of the other agents
        \vspace{1em}
        \item \textbf{Two-sided market} vs \textbf{Two-sided strategies}
        \vspace{1em}
        \item Treating two-sidedness as a market concept
    \end{itemize}
    \vspace{2em}
\quad \quad Now let's proceed to the benchmark model in Armstrong(2006)
\end{frame}

\section{Benchmark Model}
\begin{frame}{Armstrong(2006): Monopoly Pricing}
    \begin{equation*}
        \Large
        u_{j}^{i} = \alpha_{j}^{i} n^{i} + \xi_{j}^{i}
    \end{equation*}
    \begin{block}{Basic Setting}
        \begin{enumerate}
            \item Two groups of agents: j = 1, 2
            \item Only one platform in the market: i = 1
            \item No fixed benefits from using the platform: $\xi_{j}^{i} = 0$
            \item One group cares about the number of agents in the other group but not itself
            \item Interaction benefits are different across groups: $\alpha_{j}^{i} \neq \alpha_{k}^{i}$
            \item The platform can charge a per-agent fee to each group of agents
        \end{enumerate}
    \end{block}

\end{frame}

\begin{frame}{Armstrong(2006): Monopoly Pricing}
    \textbf{Utility of agents}
    \vspace{1em}
    \begin{equation}
        \large
        u_{1} = \alpha_{1} n_{2} - p_{1}
            \qquad
        u_{2} = \alpha_{2} n_{1} - p_{2}
    \end{equation}
    \begin{itemize}
      \item $\alpha_{1}$: Interaction benefit a group-1 agent gets from each group-2 agent
      \item $\alpha_{2}$: Interaction benefit a group-2 agent gets from each group-1 agent
      \item $n_{1}$: number of group-1 agents on the platform
      \item $n_{2}$: number of group-2 agents on the platform
      \item $p_{1}, p_{2}$: prices charged by the platform to groups 1 and 2
    \end{itemize}
\end{frame}

\begin{frame}{Armstrong (2006): Monopoly Pricing}
    \textbf{Demand Functions}
    \vspace{1em}
    
    \begin{equation}
        \large
        n_{1} = \phi_{1}(u_{1})
        \qquad
        n_{2} = \phi_{2}(u_{2})
    \end{equation}
    
    \begin{itemize}
      \item $n_{1}, n_{2}$: numbers of agents from groups 1 and 2 
                            who join the platform
      \item $\phi_{1}, \phi_{2}$: increasing demand functions mapping 
                                  $u_{1}, u_{2}$ (utilities) to participation
      \item $u_{1}, u_{2}$: utilities offered to agents in groups 1 and 2
    \end{itemize}
\end{frame}

\begin{frame}{Armstrong (2006): Monopoly Pricing}
    \textbf{Profit Function}
    \vspace{1em}

    \begin{equation}
        \large
        \pi(u_{1}, u_{2}) \;=\; 
            \phi_{1}(u_{1})\Bigl[\alpha_{1}\,\phi_{2}(u_{2}) - u_{1} - f_{1}\Bigr]
            \;+\;
            \phi_{2}(u_{2})\Bigl[\alpha_{2}\,\phi_{1}(u_{1}) - u_{2} - f_{2}\Bigr]
    \end{equation}
    
    \begin{itemize}
      \item $\pi(u_{1},u_{2})$: platform’s profit in terms of utilities
      \item $f_{1},f_{2}$: per‐agent costs for groups 1 and 2
      \item $\alpha_{1}, \alpha_{2}$: cross‐group interaction benefits
      \item $\phi_{1}, \phi_{2}$: same demand functions as in the previous slide
    \end{itemize}
\end{frame}

\begin{frame}{Armstrong (2006): Monopoly Pricing}
    \justifying

    Let the aggregate consumer surplus of group \(i = 1, 2\) be \(v_i(u_i)\), 
    where \(v_i(\cdot)\) satisfies the envelope condition 
    \( v_i'(u_i) \equiv \phi_i(u_i) \). Then welfare, as measured by the unweighted 
    sum of profit and consumer surplus, is
    \[
       w = \pi(u_1, u_2) \;+\; v_1(u_1) \;+\; v_2(u_2).
    \]
    
    It is easily verified that the welfare‐maximizing outcome has the utilities
    \[
       u_1 = (\alpha_1 + \alpha_2)\,n_2 \;-\; f_1,
       \qquad
       u_2 = (\alpha_1 + \alpha_2)\,n_1 \;-\; f_2.
    \]
    
    From expression (1), the socially optimal prices satisfy
    \[
       p_1 = f_1 \;-\; \alpha_2\,n_2,
       \qquad
       p_2 = f_2 \;-\; \alpha_1\,n_1.
    \]
\end{frame}

\begin{frame}{Armstrong (2006): Monopoly Pricing}
    \justifying

    From expression~(2), the profit‐maximizing prices satisfy:
    \[
      p_{1} 
      \;=\; f_{1} \;-\; \alpha_{2}\,n_{2}
              \;+\; \frac{\phi_{1}(u_{1})}{\phi_{1}'(u_{1})},
      \quad
      p_{2} 
      \;=\; f_{2} \;-\; \alpha_{1}\,n_{1}
              \;+\; \frac{\phi_{2}(u_{2})}{\phi_{2}'(u_{2})}.
    \]
\end{frame}

\begin{frame}{Armstrong (2006): Monopoly Pricing}
    \justifying  % Requires \usepackage{ragged2e} in your preamble, if desired
    
    \textbf{Proposition 1.} Write
    \[
      \eta_{1}\bigl(p_{1} \mid n_{2}\bigr) 
      \;=\; 
      \frac{p_{1}\,\phi_{1}'(\alpha_{1}n_{2} - p_{1})}{\phi_{1}(\alpha_{1}n_{2} - p_{1})},
      \quad
      \eta_{2}\bigl(p_{2} \mid n_{1}\bigr) 
      \;=\;
      \frac{p_{2}\,\phi_{2}'(\alpha_{2}n_{1} - p_{2})}{\phi_{2}(\alpha_{2}n_{1} - p_{2})}.
    \]
    
    for a group’s price elasticity of demand given the other group’s participation level.  
    Then the profit‐maximizing pair of prices satisfy
    
    \[
      \frac{p_{1} \;-\; \bigl(f_{1} - \alpha_{2}n_{2}\bigr)}{p_{1}}
      \;=\;
      \frac{1}{\eta_{1}(p_{1} \mid n_{2})},
      \quad
      \frac{p_{2} \;-\; \bigl(f_{2} - \alpha_{1}n_{1}\bigr)}{p_{2}}
      \;=\;
      \frac{1}{\eta_{2}(p_{2} \mid n_{1})}.
    \]
    
    \end{frame}

\begin{frame}{Armstrong (2006): Main Conclusions of Monopoly Pricing}
    \justifying  % Requires \usepackage{ragged2e} in the preamble
    
    A monopolist platform, serving two distinct groups that benefit from interacting 
    with one another, does not simply set each group’s price to cover the marginal 
    cost of serving that group. Instead, it balances the prices charged to each side 
    to account for cross‐group externalities and differences in demand elasticities.
    This leads to the following key insights:
    
    \begin{itemize}
        \item \textbf{Cross‐Group Externalities:} Prices on one side depend on how 
        that side’s participation affects the other side’s willingness to join.
    
        \item \textbf{Balancing Demand:} If one side’s presence is highly valuable 
        to the other side, the platform may lower or subsidize that side’s fee 
        to maximize overall profit.
    
        \item \textbf{Interdependence of Pricing:} Optimal monopoly prices for each 
        group differ from the standard one‐sided monopoly outcome, reflecting 
        the platform’s internalization of cross‐group effects.
    \end{itemize}
    
\end{frame}

\begin{frame}{Armstrong (2006): Extension of Monopoly Pricing}
    \justifying  % Requires \usepackage{ragged2e} in the preamble
    
    A monopolist platform, serving two distinct groups that benefit from interacting 
    with one another, does not simply set each group’s price to cover the marginal 
    cost of serving that group. Instead, it balances the prices charged to each side 
    to account for cross‐group externalities and differences in demand elasticities.
    This leads to the following key insights:
    
    \begin{itemize}
        \item \textbf{Cross‐Group Externalities:} Prices on one side depend on how 
        that side’s participation affects the other side’s willingness to join.
    
        \item \textbf{Balancing Demand:} If one side’s presence is highly valuable 
        to the other side, the platform may lower or subsidize that side’s fee 
        to maximize overall profit.
    
        \item \textbf{Interdependence of Pricing:} Optimal monopoly prices for each 
        group differ from the standard one‐sided monopoly outcome, reflecting 
        the platform’s internalization of cross‐group effects.
    \end{itemize}
    
\end{frame}

\begin{frame}{Armstrong(2006): Competition for the Market}
    \begin{block}{Basic Setting}
        \begin{enumerate}
            \item Two groups of agents: 1, 2
            \item Two platforms in the market: i, j
            \item No fixed benefits from using the platform: $\xi_{j}^{i} = 0$
            \item One group cares about the number of agents in the other group but not itself
            \item Interaction benefits are different across groups but same across platforms: $\alpha_{1}^{i} = \alpha_{2}^{i}, \alpha_{1}^{j} = \alpha_{2}^{j}$
            \item The platform can charge a per-agent fee to each group of agents
            \item Each agent chooses to join a single platfom [\textcolor{red}{single-homing}]
        \end{enumerate}
    \end{block}
\end{frame}

\begin{frame}{Armstrong(2006): Competition for the Market}

    \textbf{Basic model.} There are two groups of agents, labeled 1 and 2, and there are two platforms, \(A\) and \(B\), 
    which enable these groups to interact. Groups 1 and 2 obtain the respective utilities 
    \(\bigl\{u_1^i,\,u_2^i\bigr\}\) if they join platform \(i\). These utilities are determined 
    similarly to the monopoly model in (1): if platform \(i\) attracts \(n_1^i\) and \(n_2^i\) 
    members of the two groups, then the utilities on platform \(i\) are
    
    \[
       u_1^i \;=\;\alpha_1\,n_2^i \;-\; p_1^i,
       \qquad
       u_2^i \;=\;\alpha_2\,n_1^i \;-\; p_2^i.
       \tag{5}
    \]
\end{frame}
\begin{frame}{Armstrong(2006): Competition for the Market}
    When group~1 is offered a choice of utilities \(u_{1}^{A}\) and \(u_{1}^{B}\) from the two 
    platforms, and group~2 is offered the choice \(u_{2}^{A}\) and \(u_{2}^{B}\), suppose the 
    number of each group who join platform \(i\) is given by the following Hotelling specification:
    
    \[
       n_{1}^{i} 
       \;=\; 
       \tfrac{1}{2} 
       \;+\; 
       \frac{\,u_{1}^{i} \;-\; u_{1}^{j}\,}{2\,t_{1}},
       \qquad
       n_{2}^{i} 
       \;=\; 
       \tfrac{1}{2} 
       \;+\; 
       \frac{\,u_{2}^{i} \;-\; u_{2}^{j}\,}{2\,t_{2}},
       \tag{6}
    \]
    
    Here, agents in each group are assumed to be uniformly located along a unit interval, with the 
    two platforms at the endpoints. The parameters \(t_{1}, t_{2} > 0\) capture the extent of product 
    differentiation (or “transport costs”) for the two groups, describing how competitive the 
    two‐sided market is.
\end{frame}


\begin{frame}{Armstrong(2006): Competition for the Market}
    Putting (6) together with (5), and using the fact that \(n_{1}^{j} = 1 - n_{1}^{i}\),
    gives the following implicit expressions for market shares:
    \[
      n_{1}^{i} 
      \;=\; 
      \tfrac{1}{2}
      \;+\; 
      \frac{\alpha_{1}\,\bigl(2\,n_{2}^{i}\;-\;1\bigr)\;-\;\bigl(p_{1}^{i}\;-\;p_{1}^{j}\bigr)}
           {2\,t_{1}},
      \qquad
      n_{2}^{i} 
      \;=\; 
      \tfrac{1}{2}
      \;+\; 
      \frac{\alpha_{2}\,\bigl(2\,n_{1}^{i}\;-\;1\bigr)\;-\;\bigl(p_{2}^{i}\;-\;p_{2}^{j}\bigr)}
           {2\,t_{2}}
    \]
    
    Keeping its group‐2 price fixed, expression~(7) shows that an extra group‐1 agent on a platform 
    attracts a further \(\alpha_{2}/t_{2}\) group‐2 agents to that platform.
    
\end{frame}

\begin{frame}{Armstrong(2006): Competition for the Market}
    Suppose platforms \(A\) and \(B\) offer the respective price pairs 
    \(\bigl(p_{1}^{A},\,p_{2}^{A}\bigr)\) and \(\bigl(p_{1}^{B},\,p_{2}^{B}\bigr)\). 
    Given these prices, solving the simultaneous equations (7) implies that 
    the market shares are:
    
    \[
      n_{1}^{i} 
      \;=\;
      \tfrac{1}{2}
      \;+\;
      \tfrac{1}{2}\,
      \frac{
        \alpha_{1}\,\bigl(p_{2}^{j} - p_{2}^{i}\bigr)
        \;+\; 
        t_{2}\,\bigl(p_{1}^{j} - p_{1}^{i}\bigr)
      }{
        t_{1}\,t_{2} \;-\; \alpha_{1}\,\alpha_{2}
      },
      \quad
      n_{2}^{i} 
      \;=\; 
      \tfrac{1}{2}
      \;+\;
      \tfrac{1}{2}\,
      \frac{
        \alpha_{2}\,\bigl(p_{1}^{j} - p_{1}^{i}\bigr)
        \;+\;
        t_{1}\,\bigl(p_{2}^{j} - p_{2}^{i}\bigr)
      }{
        t_{1}\,t_{2} \;-\; \alpha_{1}\,\alpha_{2}
      }
    \]
\end{frame}

\begin{frame}{Armstrong(2006): Competition for the Market}
    As with the monopoly model, suppose each platform has a per‐agent cost 
    \(f_{1}\) for serving group~1 and \(f_{2}\) for serving group~2.  Then 
    platform \(i\)’s profit is:
    \begin{equation*}
        \scriptsize
        \bigl(p_{1}^{i} - f_{1}\bigr)
        \times
        \left[
           \tfrac{1}{2} 
           + 
           \tfrac{1}{2}\,\frac{\alpha_{1}\,\bigl(p_{2}^{j}-p_{2}^{i}\bigr) 
                       + t_{2}\,\bigl(p_{1}^{j}-p_{1}^{i}\bigr)}
                      {t_{1}\,t_{2}-\alpha_{1}\,\alpha_{2}}
        \right]
        +
        \bigl(p_{2}^{i} - f_{2}\bigr)
        \times
        \left[
           \tfrac{1}{2} 
           +
           \tfrac{1}{2}\,\frac{\alpha_{2}\,\bigl(p_{1}^{j}-p_{1}^{i}\bigr)
                       + t_{1}\,\bigl(p_{2}^{j}-p_{2}^{i}\bigr)}
                      {t_{1}\,t_{2}-\alpha_{1}\,\alpha_{2}}
        \right]
    \end{equation*}

    
    If \((8)\) is satisfied, one can show that no asymmetric equilibria exist, and in a 
    \emph{symmetric equilibrium} where each platform offers the same price pair 
    \(\bigl(p_{1},\,p_{2}\bigr)\), the first‐order conditions yield:
    
    \[
      p_{1} 
      =
      \underbrace{f_{1}}_{\text{cost}}
      +
      \underbrace{t_{1}}_{\text{market power}}
      -
      \underbrace{\Bigl(\tfrac{\alpha_{2}}{t_{2}}\Bigr)}_{\text{extra group‐2 agents}}
      \times
      \underbrace{\bigl(\alpha_{1} + p_{2} - f_{2}\bigr)}_{\text{profit from an extra group‐2 agent}}
      \tag{7}
    \]

\end{frame}

\begin{frame}{Armstrong(2006): Competition for the Market}
    \justifying  % Requires \usepackage{ragged2e} if you want fully justified text

    \textbf{Alternative tariffs. Uniform prices.} 
    
    \medskip
    
    Suppose \(\,f_{1} = f_{2} = f\,.\)  It makes little sense to discuss price discrimination if 
    the costs differ significantly across groups.  Assume each platform cannot set different prices 
    for groups~1 and~2, so platform \(i\) chooses a single uniform price \(p^{i}\). 
    (Perhaps sex discrimination laws prevent differential pricing by nightclubs.)  
    Then platform \(i\)’s profit is \(\bigl(p^{i} - f\bigr)\,\bigl(n_{1}^{i} + n_{2}^{i}\bigr)\).  
    From \((9)\), total demand for platform~\(i\) is
    \[
      n_{1}^{i} + n_{2}^{i}
      \;=\;
      1
      \;+\;
      \frac{1}{2}
      \frac{\,t_{1} + t_{2} + \alpha_{1} + \alpha_{2}\,}
           {\,t_{1}\,t_{2} - \alpha_{1}\,\alpha_{2}\,}
      \bigl(p^{j} - p^{i}\bigr).
    \]
    
    Solving the resulting first‐order conditions implies the equilibrium uniform price is
    \[
      p 
      \;=\; 
      f 
      \;+\;
      2\,\frac{\;t_{1}\,t_{2} \;-\; \alpha_{1}\,\alpha_{2}\;}
                {\;t_{1} \;+\; t_{2} \;+\; \alpha_{1} \;+\; \alpha_{2}\,}
      \tag{8}
    \]
\end{frame}

\begin{frame}{Armstrong(2006): Competition for the Market}
    \justifying  % Requires \usepackage{ragged2e} if you want fully justified text
    \textbf{Two-Part Tariffs}: a continuum of symmetric equilibria exist. Let 
    \(0 \,\leq\, \gamma_{1} \,\leq\, 2\alpha_{1}\) and 
    \(0 \,\leq\, \gamma_{2} \,\leq\, 2\alpha_{2}\)
    be the marginal prices charged to group~1 and group~2, respectively. 
    An equilibrium exists in which both platforms offer the same pair of two‐part tariffs 
    \(\bigl(T_{1},T_{2}\bigr)\) defined by:
    \[
       T_{1} \;=\; p_{1} \;+\; \gamma_{1}\,n_{2}, 
       \qquad
       T_{2} \;=\; p_{2} \;+\; \gamma_{2}\,n_{1},
    \]
    where the \emph{fixed fees} \(p_{1}, p_{2}\) satisfy
    \[
       p_{1} 
       \;=\; 
       f_{1} 
       \;+\; 
       t_{1} 
       \;-\; 
       \alpha_{2}
       \;+\; 
       \tfrac{1}{2}\,\bigl(\gamma_{2} - \gamma_{1}\bigr),
       \qquad
       p_{2} 
       \;=\; 
       f_{2} 
       \;+\;
       t_{2} 
       \;-\; 
       \alpha_{1}
       \;+\; 
       \tfrac{1}{2}\,\bigl(\gamma_{1} - \gamma_{2}\bigr).
       \tag{17}
    \]

    \[
       \pi 
       \;=\; 
       \frac{\,t_{1} + t_{2} - \alpha_{1} - \alpha_{2}\,}{2} 
       \;+\; 
       \frac{\gamma_{1} + \gamma_{2}}{4}.
    \]
\end{frame}

\begin{frame}{Armstrong (2006): Main Conclusions of Competition for the Market}
    \justifying  % Requires \usepackage{ragged2e} in the preamble
    
    A monopolist platform, serving two distinct groups that benefit from interacting 
    with one another, does not simply set each group’s price to cover the marginal 
    cost of serving that group. Instead, it balances the prices charged to each side 
    to account for cross‐group externalities and differences in demand elasticities.
    This leads to the following key insights:
    
    \begin{itemize}
        \item \textbf{Cross‐Group Externalities:} Prices on one side depend on how 
        that side’s participation affects the other side’s willingness to join.
    
        \item \textbf{Balancing Demand:} If one side’s presence is highly valuable 
        to the other side, the platform may lower or subsidize that side’s fee 
        to maximize overall profit.
    
        \item \textbf{Interdependence of Pricing:} Optimal monopoly prices for each 
        group differ from the standard one‐sided monopoly outcome, reflecting 
        the platform’s internalization of cross‐group effects.
    \end{itemize}
    
\end{frame}

\begin{frame}{Armstrong (2006): Extension of Competition for the Market}
    \justifying  % Requires \usepackage{ragged2e} in the preamble
    
    A monopolist platform, serving two distinct groups that benefit from interacting 
    with one another, does not simply set each group’s price to cover the marginal 
    cost of serving that group. Instead, it balances the prices charged to each side 
    to account for cross‐group externalities and differences in demand elasticities.
    This leads to the following key insights:
    
    \begin{itemize}
        \item \textbf{Cross‐Group Externalities:} Prices on one side depend on how 
        that side’s participation affects the other side’s willingness to join.
    
        \item \textbf{Balancing Demand:} If one side’s presence is highly valuable 
        to the other side, the platform may lower or subsidize that side’s fee 
        to maximize overall profit.
    
        \item \textbf{Interdependence of Pricing:} Optimal monopoly prices for each 
        group differ from the standard one‐sided monopoly outcome, reflecting 
        the platform’s internalization of cross‐group effects.
    \end{itemize}
    
\end{frame}

\begin{frame}{Armstrong(2006): Competition on the Market}
    \begin{block}{Basic Setting}
        \begin{enumerate}
            \item Two groups of agents: 1, 2
            \item Two platforms in the market: i, j
            \item No fixed benefits from using the platform: $\xi_{j}^{i} = 0$
            \item One group cares about the number of agents in the other group but not itself
            \item Interaction benefits are different across groups but same across platforms: $\alpha_{1}^{i} = \alpha_{2}^{i}, \alpha_{1}^{j} = \alpha_{2}^{j}$
            \item The platform can charge a per-agent fee to each group of agents
            \item Each agent chooses to join a single platfom [\textcolor{red}{single-homing}]
        \end{enumerate}
    \end{block}
\end{frame}

\begin{frame}{Armstrong(2006): Competition on the Market}
    \justifying  % Requires \usepackage{ragged2e} if you want fully justified paragraphs

    \textbf{A general framework.} Suppose there are two (possibly asymmetric) platforms 
    that facilitate interaction between two groups of agents.  Suppose that group‐2 
    agents are heterogeneous: if there are \(n_{1}^{i}\) group‐1 agents on platform~\(i\), 
    the number of group‐2 agents prepared to pay a fixed fee \(p_{2}^{i}\) to join that 
    platform is
    \[
       n_{2}^{i} 
       \;=\; 
       \phi^{\,i}\bigl(n_{1}^{i},\,p_{2}^{i}\bigr)
    \]
    where \(\phi^{\,i}\) is decreasing in \(p_{2}^{i}\) and increasing in \(n_{1}^{i}\). 
    A group‐2 agent’s decision to join one platform does not depend on whether she also 
    joins the rival platform.
    
    \medskip
    
    Let \(R^{i}\bigl(n_{1}^{i},\,n_{2}^{i}\bigr)\) denote platform~\(i\)’s revenue from 
    group~2 when it has \(n_{1}^{i}\) group‐1 agents and sets its group‐2 price so that 
    \(n_{2}^{i}\) group‐2 agents join. Formally,
    \[
       R^{i}\Bigl(n_{1}^{i},\,\phi^{\,i}\bigl(n_{1}^{i},\,p_{2}^{i}\bigr)\Bigr)
       \;=\;
       p_{2}^{i}\,\phi^{\,i}\bigl(n_{1}^{i},\,p_{2}^{i}\bigr)
    \]
    
    \[
       u_{1}^{i}
       \;=\;
       U^{\,i}\bigl(n_{2}^{i}\bigr)
       \;-\;
       p_{1}^{i}.
    \]
\end{frame}

\begin{frame}{Armstrong(2006): Competition on the Market}
    \justifying  % Requires \usepackage{ragged2e} if you want fully justified text

    When a group‐1 agent’s utility with platform \(i\) is \(u_{1}^{i}\), suppose the platform 
    attracts 
    \[
       n_{1}^{i} 
       \;=\; 
       \Phi^{\,i}\bigl(u_{1}^{i},\,u_{1}^{j}\bigr)
    \]
    where \(\Phi^{\,i}\) is increasing in its first argument and decreasing in its second.  
    Let \(C^{i}(n_{1}^{i},n_{2}^{i})\) denote the total cost to platform~\(i\) of serving both groups.  
    Then platform~\(i\)’s profit is
    \[
       \pi^{i}
       \;=\;
       n_{1}^{i}\,p_{1}^{i}
       \;+\;
       R^{i}\!\bigl(n_{1}^{i},\,n_{2}^{i}\bigr)
       \;-\;
       C^{i}\!\bigl(n_{1}^{i},\,n_{2}^{i}\bigr)
    \]
    where \(R^{i}\) is platform~\(i\)’s revenue from group~2 (as defined previously).
 
    
\end{frame}

\begin{frame}{Armstrong(2006): Competition on the Market}
    Next, in equilibrium, the number of group‐2 agents on each platform is derived from 
    the equilibrium market shares for group~1.  Suppose platform~\(i\) sets a group‐1 
    utility \(\hat{u}_{1}^{i}\) and thus attracts \(\hat{n}_{1}^{i}\) agents.  
    By varying \(p_{1}^{i}\) and \(n_{2}^{i}\) so that \(\hat{u}_{1}^{i} 
    = U^{\,i}\bigl(n_{2}^{i}\bigr) - p_{1}^{i}\) remains constant, the platform’s profit 
    becomes
    \[
       \pi^{i}
       \;=\;
       \hat{n}_{1}^{i}\,\Bigl[\,
         U^{\,i}\bigl(n_{2}^{i}\bigr)
         \;-\;
         \hat{u}_{1}^{i}
       \Bigr]
       \;+\;
       R^{i}\!\bigl(\hat{n}_{1}^{i},\,n_{2}^{i}\bigr)
       \;-\;
       C^{i}\!\bigl(\hat{n}_{1}^{i},\,n_{2}^{i}\bigr).
    \]
    Given \(\hat{n}_{1}^{i}\), platform~\(i\) chooses \(\hat{n}_{2}^{i}\) to maximize 
    \(\hat{n}_{1}^{i}\,U^{\,i}(\cdot) + R^{i}\bigl(\hat{n}_{1}^{i},\cdot\bigr) 
    - C^{i}\bigl(\hat{n}_{1}^{i},\cdot\bigr)\), i.e.,
    \[
       \hat{n}_{2}^{i}
       \;=\;
       \arg\max_{n_{2}}
       \bigl[\,
          \hat{n}_{1}^{i}\,U^{\,i}(n_{2})
          \;+\;
          R^{i}\!\bigl(\hat{n}_{1}^{i},\,n_{2}\bigr)
          \;-\;
          C^{i}\!\bigl(\hat{n}_{1}^{i},\,n_{2}\bigr)
       \bigr]
    \]
    The equilibrium price to group~2 is \(\hat{p}_{2}^{\,i}\), where
    \[
       \hat{n}_{2}^{i}
       \;=\;
       \phi^{\,i}
       \bigl(\hat{n}_{1}^{i},\,\hat{p}_{2}^{\,i}\bigr)
    \]
    
\end{frame}

\begin{frame}{Armstrong (2006): Main Conclusions of Competition on the Market}
    \justifying  % Requires \usepackage{ragged2e} in the preamble
    
    A monopolist platform, serving two distinct groups that benefit from interacting 
    with one another, does not simply set each group’s price to cover the marginal 
    cost of serving that group. Instead, it balances the prices charged to each side 
    to account for cross‐group externalities and differences in demand elasticities.
    This leads to the following key insights:
    
    \begin{itemize}
        \item \textbf{Cross‐Group Externalities:} Prices on one side depend on how 
        that side’s participation affects the other side’s willingness to join.
    
        \item \textbf{Balancing Demand:} If one side’s presence is highly valuable 
        to the other side, the platform may lower or subsidize that side’s fee 
        to maximize overall profit.
    
        \item \textbf{Interdependence of Pricing:} Optimal monopoly prices for each 
        group differ from the standard one‐sided monopoly outcome, reflecting 
        the platform’s internalization of cross‐group effects.
    \end{itemize}
    
\end{frame}

\begin{frame}{Armstrong (2006): Extension of Competition on the Market}
    \justifying  % Requires \usepackage{ragged2e} in the preamble
    
    A monopolist platform, serving two distinct groups that benefit from interacting 
    with one another, does not simply set each group’s price to cover the marginal 
    cost of serving that group. Instead, it balances the prices charged to each side 
    to account for cross‐group externalities and differences in demand elasticities.
    This leads to the following key insights:
    
    \begin{itemize}
        \item \textbf{Cross‐Group Externalities:} Prices on one side depend on how 
        that side’s participation affects the other side’s willingness to join.
    
        \item \textbf{Balancing Demand:} If one side’s presence is highly valuable 
        to the other side, the platform may lower or subsidize that side’s fee 
        to maximize overall profit.
    
        \item \textbf{Interdependence of Pricing:} Optimal monopoly prices for each 
        group differ from the standard one‐sided monopoly outcome, reflecting 
        the platform’s internalization of cross‐group effects.
    \end{itemize}
    
\end{frame}

\section{Extensions and Insights}
\begin{frame}{Results}
    \begin{columns}
        \column{0.5\textwidth}
        \begin{itemize}
            \item Result 1
            \item Result 2
        \end{itemize}
        
        \column{0.5\textwidth}
        % Placeholder for a figure
        \centering
        [Your figure here]
    \end{columns}
\end{frame}

% Section 4
\section{Conclusion and Future Directions}
\begin{frame}{Conclusion and Future Directions}
    \justifying  % Requires \usepackage{ragged2e} if you want full justification
    \textbf{Monopoly and Competition.}
    \begin{itemize}
      \item Monopoly Platforms: Compared profit‐maximizing versus welfare‐maximizing 
            outcomes, showing how network effects influence price levels.
      \item Platform Competition: Discussed how modeling “competition for the market” versus 
            “competition on the market” must account for single‐homing or multi‐homing. 
            Also emphasized matching design and within‐side price discrimination.
    \end{itemize}
\end{frame}

\begin{frame}
    \textbf{Gaps and Future Research.}
    \begin{itemize}
      \item \emph{Dynamic Aspects:} Literature remains largely static, with limited work on ignition 
            or evolving platform competition.
      \item \emph{Multi‐Homing Nuances:} Different values for agents’ interactions across sides 
            require richer models of multi‐homing behavior.
      \item \emph{Discrimination and Design:} Need more flexible models covering discriminatory 
            practices, bundling, feedback/recommendation systems, integration, and entry rules.
      \item \emph{Industry‐Specific Studies:} Areas like media, finance, housing, transportation, 
            and health insurance offer promising directions for deeper analysis and policy implications.
    \end{itemize}
\end{frame}
\end{document}